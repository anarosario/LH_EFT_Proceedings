%= local definitions of macros ============================
\newcommand{\Herwig}{H\protect\scalebox{0.8}{ERWIG}\xspace}
\newcommand{\Pythia}{P\protect\scalebox{0.8}{YTHIA}\xspace}
\newcommand{\Sherpa}{S\protect\scalebox{0.8}{HERPA}\xspace}
\newcommand{\Rivet}{R\protect\scalebox{0.8}{IVET}\xspace}
\newcommand{\Professor}{P\protect\scalebox{0.8}{ROFESSOR}\xspace}
\newcommand{\eps}{\varepsilon}
\newcommand{\mc}[1]{\mathcal{#1}}
\newcommand{\mr}[1]{\mathrm{#1}}
\newcommand{\mb}[1]{\mathbb{#1}}
\newcommand{\tm}[1]{\scalebox{0.95}{$#1$}}
\newcommand{\SMEFTsim}{\texttt{SMEFTsim}}
\newcommand{\SMEFTatNLO}{\texttt{SMEFT@NLO}}
\newcommand{\Madgraph}{M\protect\scalebox{0.8}{ADGRAPH}\xspace}
\tikzset{
  particle/.style={thick, draw=black, postaction={decorate},
    decoration={markings, mark=at position 0.6 with {\arrow[black]{stealth}}}},
  virtualparticle/.style={thick, draw=black, postaction={decorate}},
  scalar/.style={densely dashed, thick, draw=black, postaction={decorate}},
  photon/.style={decorate, draw=black,
    decoration={coil, aspect=0, amplitude=3pt, segment length=6pt}},
  gluon/.style={decorate, draw=black,
    decoration={coil, segment length=5pt, amplitude=4pt}}
}

\newcommand{\cH}{\ensuremath{c_{H}}\xspace}
\newcommand{\cHbox}{\ensuremath{c_{H\square}}\xspace}
\newcommand{\cHDD}{\ensuremath{c_{HDD}}\xspace}
\newcommand{\cHG}{\ensuremath{c_{HG}}\xspace}
\newcommand{\cHW}{\ensuremath{c_{HW}}\xspace}
\newcommand{\cHB}{\ensuremath{c_{HB}}\xspace}
\newcommand{\cHWB}{\ensuremath{c_{HWB}}\xspace}
\newcommand{\ceHAbs}{\ensuremath{|c_{eH}|}\xspace}
\newcommand{\cuHAbs}{\ensuremath{|c_{uH}|}\xspace}
\newcommand{\cdHAbs}{\ensuremath{|c_{dH}|}\xspace}
\newcommand{\ceWAbs}{\ensuremath{|c_{eW}|}\xspace}
\newcommand{\ceBAbs}{\ensuremath{|c_{eB}|}\xspace}
\newcommand{\cuGAbs}{\ensuremath{|c_{uG}|}\xspace}
\newcommand{\cuWAbs}{\ensuremath{|c_{uW}|}\xspace}
\newcommand{\cuBAbs}{\ensuremath{|c_{uB}|}\xspace}
\newcommand{\cdGAbs}{\ensuremath{|c_{dG}|}\xspace}
\newcommand{\cdWAbs}{\ensuremath{|c_{dW}|}\xspace}
\newcommand{\cdBAbs}{\ensuremath{|c_{dB}|}\xspace}
\newcommand{\cHl}[1]{\ensuremath{c_{Hl#1}}\xspace}
\newcommand{\cHe}{\ensuremath{c_{He}}\xspace}
\newcommand{\cHq}[1]{\ensuremath{c_{Hq#1}}\xspace}
\newcommand{\cHu}{\ensuremath{c_{Hu}}\xspace}
\newcommand{\cHd}{\ensuremath{c_{Hd}}\xspace}

\newcommand{\cpDC}{\ensuremath{c_{pDC}}\xspace}
\newcommand{\cpG}{\ensuremath{c_{pG}}\xspace}
\newcommand{\cdp}{\ensuremath{c_{dp}}\xspace}
\newcommand{\cpe}{\ensuremath{c_{pe}}\xspace}
\newcommand{\cpl}[1]{\ensuremath{c_{pl#1}}\xspace}
\newcommand{\cpmu}{\ensuremath{c_{pmu}}\xspace}
\newcommand{\cpqi}{\ensuremath{c_{pq3i}}\xspace}
\newcommand{\ctpl}[1]{\ensuremath{c_{3pl#1}}\xspace}
\newcommand{\cpd}{\ensuremath{c_{pd}}\xspace}
\newcommand{\cpQ}{\ensuremath{c_{pQ3}}\xspace}
\newcommand{\cpQM}{\ensuremath{c_{pQM}}\xspace}
\newcommand{\cpqMi}{\ensuremath{c_{pqMi}}\xspace}
\newcommand{\cpt}{\ensuremath{c_{pt}}\xspace}
\newcommand{\cpu}{\ensuremath{c_{pu}}\xspace}
\newcommand{\ctG}{\ensuremath{c_{tG}}\xspace}
\newcommand{\ctp}{\ensuremath{c_{tp}}\xspace}
\newcommand{\cpW}{\ensuremath{c_{pW}}\xspace}
\newcommand{\cpBB}{\ensuremath{c_{pBB}}\xspace}
\newcommand{\cpWB}{\ensuremath{c_{pWB}}\xspace}
\newcommand{\ctB}{\ensuremath{c_{tB}}\xspace}
\newcommand{\ctW}{\ensuremath{c_{tW}}\xspace}



%= title + authors =====================================
\section{Study of EFT effects in loop induced Higgs processes ~\protect\footnote{
  A.~Cueto,
  S.~Pigazzini}{}}

%= MANDATORY label ======================================
\label{sec:projname}

%= (optional) preamble ================================== 
%\begin{abstract}

%\end{abstract}

%= content ===== ========================================
\subsection{Introduction}
\label{sec:higgseft:section1}
The Standard Model Effective Field Theory (SMEFT) approach is a powerful tool to look for hints of new physics. It allows to study large sets of experimental data without assuming that the theory used is valid to arbitrarily high energies. In the SMEFT, the Standard Model (SM) as we know it is just an effective theory at energies around the electroweak scale. Beyond the Standard Model (BSM) physics manifests at higher scales, $\Lambda$, and is parameterised in terms of higher-dimmensional operators that conserve the same fields and symmetries as the SM. At any mass dimension, a complete bases of non-reduntant operators can be worked out and the full Lagrangian can be written as a power expansion
\begin{equation}
\mathcal{L}_{\textrm SMEFT} = \mathcal{L}_{\textrm SM} + \sum_{d>4}\sum_{i}\frac{c_i}{\Lambda^{d-4}}\mathcal{O}_{i}^{(d)},
\end{equation}  

where $\mathcal{L}_{\textrm SM}$ is the SM Lagrangian, $c_i$ are the Wilson coefficients and ${\mathcal{O}^{d}}$ the set of independent operators for dimension $d$. Operators with $d=5,7$  violate lepton and/or baryon number conservation~\cite{Degrande:2012wf,Kobach:2016ami}. Thus, dimension-6 operators represent the leading deviation from the SM and will be the focus of this work. The modification of a cross section by the insertion of one dimesion-6 operator in the amplitudes can be written as

\begin{equation}
\sigma = \sigma_{\textrm SM} + \sum_{i}\sigma_i^{\textrm int} \frac{c_i}{\Lambda^{2}} + \sum_{i,j}\sigma_{(i,j)}^{\textrm BSM} \frac{c_ic_j}{\Lambda^{4}},
\end{equation}  

where $\sigma_{\textrm SM}$ is the SM cross section of a given process, $\sigma_i^{\textrm int}$ is the interference between the SM and BSM amplitudes and $\sigma_{(i,j)}^{\textrm BSM}$ represents the pure BSM correction to the SM cross section. The leading term is formally $\sigma_i^{\textrm int}$ and the one than will be investigated in this work. 

Several bases of independent operators can be found in the literature~\cite{Grzadkowski:2010es,Contino:2013kra,Gupta:2014rxa,Masso:2014xra}. In the context of the study of the Higgs boson, the SILH basis~\cite{Contino:2013kra} has been commonly used. However, it is not optimised for, for example, diboson processes. Even if the translation between bases is known and has been automated~\cite{Falkowski:2015wza,Aebischer:2017ugx}, experimental collaboration have started to publish their EFT interpretations in the Warsaw basis also in the Higgs sector~\cite{ATLAS:2019jst,ATL-PHYS-PUB-2019-042} to facilitate future global fits of electroweak, Higgs and top data.

The procedure to test the EFT effects for a given set of measurements can be tedious in practice and a big effort has been devoted to develop public code to perform this task in a automatic and generic way~\cite{Brivio:2019irc}. For the Warsaw basis, different Universal FeynRules Output (UFO)~\cite{Degrande:2011ua} models are available which can be interfaced with modern event generators.

The \SMEFTsim\ code~\cite{Brivio:2017btx} is a well documented UFO implementation of the full set of dimension-6 operators in the Warsaw basis. Its main scope is the estimation of the leading SMEFT corrections to the SM. The effective Lagrangian is truncated at $\Lambda^{-2}$ and not supported for next-to-leading-order (NLO) simulations. For Higgs data interpretation the model have become of common use due to its completeness~\cite{Ellis:2018gqa,ATLAS:2019jst}.  To reproduce all the main Higgs production and decay channels in the SM, the loop-induced processes ($hgg$, $h\gamma\gamma$,$hZ\gamma$) are included as effective vertices. However, this implementation might not result satisfactory for reasons as the ones exposed below:
\begin{itemize}
\item Only operators with the same point-like structure as the effective vertices included to reproduce loop-induced processes can modify the cross sections of these processes. That means that, for example, a modification of the top Yukawa will not affect the gluoon-gluon fusion Higgs production process.
\item Given the truncation of the Lagrangian, operators that enter in the shifts to input parameters and that will modify the cross section of any tree-level process does not modify the cross section of loop-induced processes.
\item A reliable computation of the Higgs plus jet  production in gluon-gluon fusion requires top quark loop amplitudes at high $p_{\textrm T}$ and the implementation of $gggH$ vertices.
\item The $gg\to ZH$ process cannot be simulated.
\end{itemize}

To overcome these concerns the \SMEFTatNLO\ tool~\cite{SMEFTNLO} can be used for the loop induced Higgs processes. The tool includes a complete implemation of the SMEFT compatible with NLO QCD predictions. In this work, we study the $gg\to ZH$ and $gg\to H$ processes using this tool. 







\subsection{Comparison between models}
\label{sec:higgseft:section2}
The \SMEFTsim\ and \SMEFTatNLO\ tools have been validated against each other~\cite{Durieux:2019lnv} for the top sector. In this section, we compare both models at leading order (LO) by checking the cross sections of the $pp\to ZH$ and $pp\to t\bar{t}H$ processes. The comparison is made at the cross section level and, thus, not expected to be in perfect agreement since it will be affected by phase-space integration. The main goal of this comparison is to show the mapping between the different Wilson coefficients naming and to ensure that the setup used for both models is consistent.

For both models we use the $m_Z$, $m_W$, $G_F$ scheme of electroweak parameters\footnote{We use the \texttt{SMEFTsim\_A\_U35\_MwScheme\_UFO} model for \SMEFTsim\ and the \texttt{SMEFTatNLO\_U2\_2\_U3\_3\_cG\_4F\_LO\_UFO-LO} model for \SMEFTatNLO\ }. The latest versions of the models available in December 2019 are used.  The \Madgraph 2.6.6 generator is used to obtain the cross sections results. The definition of the ZH and ttH processes is as follows for the SM predictions:

\noindent
  \texttt{ define p = p b b$\sim$ }\\
  \texttt{ generate p p $>$ h t t$\sim$ SMHLOOP=0 NP=0 }\\
and \\ 
  \texttt{ generate p p $>$ h l+ l- SMHLOOP=0  NP=0     }.\\
  
  The values of several parameters like $m_W$, $mt$, $\alpha_S$ or $\Gamma_{H}$ differ in the default settings of the models and they were set to the same values.

  The tables below show the comparison between the predictions obtained for SM in both models as well as the interference terms, obtained with the NP$^{\wedge}$2$==$1 (NP$^{\wedge}$2$==$2)  for the \SMEFTsim\ (\SMEFTatNLO\ ) model.

  \begin{table}[h!]
    \centering
    \resizebox{\columnwidth}{!}{% 
    \begin{tabular}{|l|l|c|c|}
      \hline
      \textbf{Operator} &\textbf{W. coefficient} & \textbf{SMEFTsim} & \textbf{SMEFTatNLO} \\
      \hline 
      &  SM-SM & 0.0251$\pm$ 0.0001& 0.0255$\pm$ 0.0003\\
      \hline 
      $\partial_{\mu}(\phi^{\dag}\phi)\partial^{\mu}(\phi^{\dag}\phi)$ & \cpd\ (\cHbox) & 0.00304$\pm$ 0.00001& 0.00308 $\pm$ 0.00003\\
      \hline
      $(\phi^{\dag}D_\mu\phi)^\dag(\phi^{\dag}D_\mu\phi)  $& \cpDC\ (\cHDD) & 0.00041$\pm$ 0.00001& 0.00043$\pm$ 0.00006\\
      \hline 
      $\left(\phi^\dag\phi-\frac{v^2}{2}\right)B^{\mu\nu}B_{\mu\nu}  $ & \cpBB\ (\cHB) & 0.00231$\pm$ 0.00001& 0.00229$\pm$ 0.00004\\
      \hline
      $\left(\phi^\dag\phi-\frac{v^2}{2}\right)W_{I}^{\mu\nu}W^{I}_{\mu\nu}$ & \cpW\ (\cHW) & 0.01818$\pm$ 0.00007& 0.0183$\pm$ 0.0002\\
      \hline 
      $\left(\phi^\dag\phi-\frac{v^2}{2}\right)B^{\mu\nu}W^{I}_{\mu\nu} $ & \cpWB\ (\cHWB) & 0.00838$\pm$ 0.00004& 0.0084$\pm$ 0.0001\\
      \hline  
      $ i(\phi^{\dag}\overleftrightarrow{D}_\mu \phi)(\bar d_i \gamma^\mu d_i)$ & \cpd\ (\cHd) &-0.0044$\pm$ 0.0002 & -0.00444$\pm$ 0.00004\\
      \hline
      $ i(\phi^{\dag}\overleftrightarrow{D}_\mu \phi)(\bar{e} \gamma^\mu e)$ & \cpe\ $+$\cpmu (\cHe) &-0.002853$\pm$ 0.000007& -0.00285$\pm$ 0.00001\\
      \hline
      $i(\phi^{\dag}\overleftrightarrow{D}_\mu \phi)(\bar{l_{1,2}}\gamma^\mu l_{1,2}) $ & \cpl1\ $+$ \cpl2\ (\cHl1) & 0.00324$\pm$ 0.00002 & 0.00327$\pm$ 0.00002\\
      \hline
      $i(\phi^{\dag}\overleftrightarrow{D}_\mu \tau_{I}\phi)(\bar{l_{1,2}}\gamma^\mu \tau^{I} l_{1,2}) $ & \ctpl1+\ctpl2 (\cHl3) & -0.00588$\pm$ 0.00002& -0.00590$\pm$ 0.00005\\
      %cHq3-c3pqi & 0.049$\pm$ 0.001& 0.0478$\pm$ 0.0005\\
      %\hline
      
      \hline
    \end{tabular}
}    
    \caption{ Comparison of the SM and interference predicitions for the Z($l^{+}l^{-}$)H process between the \SMEFTsim\ and \SMEFTatNLO. The operators definitions are consistent with those given in \SMEFTatNLO. The Wilson coefficients use an analogous defintion to those provided  in the UFO model in \SMEFTatNLO\ and \SMEFTsim\ in parenthesis.}
    \label{tab:higgseft:zhcompa}
  \end{table}

\begin{table}[h!]
  \resizebox{\columnwidth}{!}{% 
    
    \begin{tabular}{|l|l|c|c|}
      \hline
      \textbf{Operator}& \textbf{W. coefficient} & \textbf{SMEFTsim} & \textbf{SMEFTatNLO} \\
      \hline
      &  SM-SM & 0.402$\pm$ 0.001& 0.402$\pm$ 0.003\\
      \hline
      $\partial_{\mu}(\phi^{\dag}\phi)\partial^{\mu}(\phi^{\dag}\phi)$  & \cpd\ (\cHbox) & 0.049$\pm$ 0.001 & 0.04876$\pm$ 0.00002\\
      \hline
      $(\phi^{\dag}D_\mu\phi)^\dag(\phi^{\dag}D_\mu\phi)$ & \cpDC\ (\cHDD) & -0.01218$\pm$ 0.00002 & -0.01222$\pm$ 0.00008\\
      \hline
      $\left(\phi^\dag\phi-\frac{v^2}{2}\right)B^{\mu\nu}B_{\mu\nu}  $ & \cpBB\ (\cHB) & 0.0000893$\pm$ 0.0000002 & 0.0000897$\pm$ 0.0000008\\
      \hline
      $\left(\phi^\dag\phi-\frac{v^2}{2}\right)W_{I}^{\mu\nu}W^{I}_{\mu\nu}$ & \cpW\ (\cHW)& 0.00042$\pm$ 0.000001& 0.000423$\pm$ 0.000004\\
      \hline
      $\left(\phi^\dag\phi-\frac{v^2}{2}\right)B^{\mu\nu}W^{I}_{\mu\nu}$ &  \cpWB\ (\cHWB)& -0.0002499$\pm$ 0.0000005& -0.000253$\pm$ 0.000002\\
      \hline
      $ i(\phi^{\dag}\overleftrightarrow{D}_\mu \phi)(\bar d_i \gamma^\mu d_i)$ & \cpd\ (\cHd) & -0.0000761$\pm$ 0.0000003 & -0.000076$\pm$ 0.000002\\
      \hline
      $\left(\phi^{\dag}\phi-\frac{v^2}{2}\right)\bar{Q}t\tilde{\phi} + h.c.$ & cuHAbs-ctp & -0.0488$\pm$ 0.0001& -0.0494$\pm$ 0.0003\\
      \hline
      $i g_{s}\left(\bar{Q}\tau^{\mu\nu}T_{A}t\right)\tilde{\phi}G_{\mu\nu}^{A} + h.c.$ & cuGAbs-ctG & -0.3393$\pm$ 0.0009 & 0.407 $\pm$ 0.002\\
      \hline
      $i(\phi^{\dag}\overleftrightarrow{D}_\mu \tau_{I}\phi)(\bar{l_{1,2}}\gamma^\mu \tau^{I} l_{1,2}) $ & \ctpl1+\ctpl2 (\cHl3) & -0.0489 $\pm$ 0.0001 & -0.0491 $\pm$ 0.0002\\
      
      % cHq3-c3pqi & 0.20976E-02$\pm$ 0.56257E-05& 0.12428E-02$\pm$ 0.17784E-04\\
      % \hline
      
      % \hline
      %$(\bar q_p \sigma^{\mu\nu} u_r) \widetilde H \, B_{\mu\nu}$ & cuBAbs-ctZ & -0.00083$\pm$ 0.000002 & -0.00175$\pm$ 0.00002\\
      % \hline
      %$(\bar q_p \sigma^{\mu\nu} u_r) \tau^I \widetilde H \, W_{\mu\nu}^I$ &cuWAbs-ctW & -0.002219$\pm$ 0.000006& 0.00378$\pm$ 0.00003\\
      % & cuBAbs-ctB & -0.82828E-03$\pm$ 0.20643E-05& 0.84754E-03$\pm$ 0.10257E-04\\
      % \hline
      % & cuWAbs-ctW & -0.22186E-02$\pm$ 0.63027E-05& 0.22314E-02$\pm$ 0.22095E-04\\
      
      \hline
    \end{tabular}
  }
  \caption{ Comparison of the SM and interference predicitions for the ttH process between the \SMEFTsim\ and \SMEFTatNLO. The operators definitions are consistent with those given in \SMEFTatNLO. The Wilson coefficients use an analogous defintion to those provided  in the UFO model in \SMEFTatNLO\ and \SMEFTsim\ in parenthesis.}
  \label{tab:higgseft:tthcompa}
\end{table}


In the tables and throughout this note, the same definitions of operators and fields as provided in~\cite{SMEFTNLOdefs} are used. In this notation, $gs$ is the strong coupling constant and $v$ denotes the vacuum expectiation value of the Higgs field $\phi$. $Q$ is the third generation left-handed quark $SU(2)$-doublet, $t$ is right handed $SU(2)$-singlet top quark. $G_{\mu\nu}^{A}$, $B_{\mu\nu}$, $W^{I}_{\mu\nu}$ are the fields strength tensors. Finally, $T^{A}$ is the generator of the fundamental representation of $SU(3)$ and $\tau^{\mu\nu}=\frac{1}{2}[\gamma^{\mu},\gamma^{\nu}]$  with $\gamma^{\mu}$ the Dirac gamma matrices.


%(In SMEFTsim notation)In the tables, $H$ is the SU$_{\textrm L}$(2) scalar doublet. The gauge covariant derivative is defined as  $D_\mu = \partial_\mu + i g_3 T^A A^A_\mu + i g_2  t^I W^I_\mu + i g_1 y B_\mu$, where $T^A$ are the $\rm SU_c(3)$ generators, $t^I=\tau^I/2$ are the $\rm SU_L(2)$ generators, and $y$ is the $\rm U_Y(1)$ hypercharge. The fields $\{q,l \}$ are left handed and the fields $\{e,u,d\}$ are right handed. The definition $\sigma_{\mu \, \nu} = i \, [\gamma_\mu,\gamma_\nu]/2$ is used. The subscripts $p,r,s$ and $t$ are flavour indices.

For the ttH production mode differences are observed for the \ctG\ operator. These differences are acknowledged by the authors of the models and reside in the absence of five-point interactions and higher in the \SMEFTsim\ model. It will be corrected in future versions of the model.

For the $\mathcal{O}_{uW}$ and $\mathcal{O}_{uB}$ operators defined as,

$$ \mathcal{O}_{tB} = i(\bar Q \sigma^{\mu\nu} t) \tilde \phi  B_{\mu\nu} + h.c.; \quad
   \mathcal{O}_{tW} =  i(\bar{Q} \tau^{\mu\nu}\tau_{I} t)  \tilde{\phi} \, W_{\mu\nu}^I + h. c.
$$

   there is no one-to-one correspondence between the models in their latest versions. The \SMEFTatNLO\ version released on 2019/04/03 was used instead to compare these two operators.

\begin{center}
  \begin{table}[h!]
    \resizebox{\columnwidth}{!}{%  
      \begin{tabular}{|l|l|c|c|}
        \hline
        \textbf{Operator}& \textbf{W. coefficient} & \textbf{SMEFTsim} & \textbf{SMEFTatNLO} \\
        \hline
        \hline                                                                                                                                              
        $i(\bar Q \sigma^{\mu\nu} t) \tilde \phi  B_{\mu\nu} + h.c.$ & \ctB\ (\cuBAbs) & -0.000828$\pm$ 0.000002 &-0.00085$\pm$ 0.00001 \\
         \hline
        $i(\bar{Q} \tau^{\mu\nu}\tau_{I} t)  \tilde{\phi} \, W_{\mu\nu}^I +h.c$ & \ctW (\cuWAbs) & -0.002219$\pm$ 0.000006& 0.00223$\pm$ 0.00002\\
        \hline
 
\end{tabular}
    }
    \caption{ Comparison of the SM and interference predicitions for the ttH process between the \SMEFTsim\ and \SMEFTatNLO\ for \ctB\ (\cuBAbs) and \ctW\ (\cuWAbs). The operator definition are given in the way they are implemented in \SMEFTatNLO\ .}
    \label{tab:higgseft:comptth2}
\end{table}
\end{center}

The prediction for the operators shown in Table~\ref{tab:higgseft:comptth2} agree in their absolute value but not in their sign. The way in which they are implemented in the model is also different. While in \SMEFTsim\ the absolute value and the phase of these complex operators can be changed by the user, only the real part can be tuned by the user in \SMEFTatNLO.

Other differences come from two-fermion operators involving quarks. In \SMEFTsim\ the couplings of all quarks enter equally, while in \SMEFTatNLO\ the top vertices are parameterized separately. 


\subsection{$gg\to Z(l^{+}l^{-})H$}
\label{sec:higgseft:ggzh}
The study of  $gg\to Z(l^{+}l^{-})H$ is performed using the \SMEFTatNLO\ model. The renormalization and factorization scales are set to $M_H=125$~GeV and the PDF set NNPDF2.3 for the parametrisation of the proton structure is used. The generated events are passed through the \Pythia\ parton shower. A more in depth study of the SMEFT effects for this process was performed in~\cite{Bylund:2016phk} using the main set of operators affecting the cross sections using merged samples of up to one additional parton. Here we have considered all the operators available at NLO in \SMEFTatNLO\ which provide diagrams with a non-zero interference with the SM.

In Figure~\ref{fig:higgseft:ggzh}, differential distributions as functions of $p_{T}^{V}$ and $m_{HV}$  with BSM effects caused by \cpqi\ , \cpu\ , \ctG\ and \ctp\ are shown.  Many other operators modify the cross section of this process but only some examples of those that distort significantly the shape of the SM prediction for  $c_i=1$ are shown.

\begin{figure}
\includegraphics[width=0.49\linewidth,page=7]{figures/kinematics_ggHll_np0.pdf}
\includegraphics[width=0.49\linewidth,page=10]{figures/kinematics_ggHll_np0.pdf}
\label{fig:higgseft:ggzh}
\caption{Differential distributions as a function of $p_{T}^{V}$ and $m_{HV}$ for the SM predictions and its interference with operators with Wilson coefficients \ctG\ , \cpd\ , \cpu\ and \ctp\  at the lowest order in QCD. The value of $\Lambda$ was set to 1 TeV}  
\end{figure}

In addition to differential cross sections, measurements  of the Higgs couplings in terms of Simplified Template Cross Sections (STXS)~\cite{deFlorian:2016spz} also provide constraining power of the SMEFT parameters. A parametrisation in bins of the STXS in stage 1.2~\cite{Berger:2019wnu}  for $gg\to Z(l^{+}l^{-})H$  is provided in Table~\ref{tab:higgseft:stxsggzh}.

\begin{table}[h!]
 \adjustbox{max width=\textwidth}{
  \begin{tabular}{p{0.40\textwidth} p{0.60\textwidth}}
    \toprule


      \hline
      \textbf{Bin} & Parametrization \\
      \hline 
    $gg\to Hll (p_{\textrm T}^{V}<75$~GeV)&
     -0.0012 \cpDC +0.121 \cdp -0.056 \cpe +0.064 \cpl1 +0.064 \cpl2 -0.0566 \cpmu -0.331 \cpqi -0.117 \ctpl1 -0.117 \ctpl2 +0.249 \cpd -0.166 \cpQ -0.129 \cpQM -0.332 \cpqMi +0.047 \cpt +0.165 \cpu +0.250 \ctG +0.0369 \ctp  \\
     \hline 
     $gg\to Hll (75<p_{\textrm T}^{V}<150$~GeV)&
     +0.0030 \cpDC +0.122 \cdp -0.057 \cpe +0.065 \cpl1 +0.065 \cpl2 -0.0568 \cpmu -0.285 \cpqi -0.118 \ctpl1 -0.118 \ctpl2 +0.213 \cpd -0.142 \cpQ -0.098 \cpQM -0.283 \cpqMi +0.0262 \cpt +0.142 \cpu +0.316 \ctG +0.0454 \ctp\\
     \hline
     $gg\to Hll $(0-jet,$150<p_{\textrm T}^{V}<250$~GeV)&
     +0.025 \cpDC +0.120 \cdp -0.057 \cpe +0.065 \cpl1 +0.065 \cpl2 -0.0561 \cpmu -0.233 \cpqi -0.116 \ctpl1 -0.118 \ctpl2 +0.17 \cpd -0.115 \cpQ -0.029 \cpQM -0.229 \cpqMi -0.027 \cpt +0.112 \cpu +0.439 \ctG +0.084 \ctp  \\
     \hline
     $gg\to Hll (\geq$ 1-jet,$150<p_{\textrm T}^{V}<250$~GeV)&
 +0.016 \cpDC +0.122 \cdp -0.0569 \cpe +0.065 \cpl1 +0.065 \cpl2 -0.0572 \cpmu -0.244 \cpqi -0.118 \ctpl1 -0.117 \ctpl2 +0.183 \cpd -0.122 \cpQ -0.050 \cpQM -0.245 \cpqMi -0.0111 \cpt +0.121 \cpu +0.411 \ctG +0.072 \ctp  \\
 \hline

      $gg\to Hll (p_{\textrm T}^{V}>250$~GeV)&
 +0.049 \cpDC +0.120 \cdp -0.0585 \cpe +0.066 \cpl1 +0.066 \cpl2 -0.0581 \cpmu -0.197 \cpqi -0.116 \ctpl1 -0.116 \ctpl2 +0.153 \cpd -0.099 \cpQ +0.031 \cpQM -0.199 \cpqMi -0.0820 \cpt +0.099 \cpu +0.544 \ctG +0.134 \ctp \\
 \bottomrule
\end{tabular}
}

\caption{Parametrization of the $gg\to ZH$ bins of the STXS as defined in its stage 1.2 with the parameters definitions of the \SMEFTatNLO\ model. The numbers are rounded according to their statistical uncertainty.}

\end{table}

\subsection{$gg\to H$}
\label{sec:higgseft:section3}
The SMEFT effects in the Higgs production through gluon-gluon fusion is examined using the \SMEFTatNLO\ package. As in Section~\ref{sec:higgseft:ggzh}, the 
study of this process is already available in the literature~\cite{Deutschmann:2017qum} for a limited set of operators. In this work we have considered all operators that have a non-zero interference with the SM. Those operators were found to be:
$\mathcal{O}_{\phi G}$, $\mathcal{O}_{tG}$, $\mathcal{O}_{t\phi}$, $\mathcal{O}_{d\phi}$, $\mathcal{O}_{\phi DC}$, $\mathcal{O}_{\phi l1}^{(3)}$ and $\mathcal{O}_{ \phi l2}^{(3)}$.
The last four operators enters in the process though shifts to the inputs parameters and dot modify the shape of the SM predictions.


 \textcolor{red}{ SETUP, check with Simone}


  
 \textcolor{red}{PLOTS}

  

 In Table~\ref{tab:higgseft:stxsggh1}, we provide the parametrisation of the $gg\to H$ STXS bins in stage 1.2.

 
 
\begin{table}[h!]
 \adjustbox{max width=\textwidth}{
  \begin{tabular}{p{0.40\textwidth} p{0.60\textwidth}}
    \toprule


      \hline
      \textbf{Bin} & Parametrization \\      
      $gg\to H$ ($200<p_{\textrm T}^{H}<300$~GeV)&
      1.7 \ctG -0.06 \ctpl1 -0.06 \ctpl2 +0.12 \cdp -0.03 \cpDC -0.12 \ctp +45 \cpG \\
      \hline
      $gg\to H$ ($300<p_{\textrm T}^{H}<450$~GeV)&
      1.9 \ctG -0.06 \ctpl1 -0.06 \ctpl2 +0.12 \cdp -0.03 \cpDC -0.12 \ctp +50 \cpG \\
      \hline
      $gg\to H$ ($450<p_{\textrm T}^{H}<650$~GeV)&
      2.5 \ctG -0.06 \ctpl1 -0.06 \ctpl2 +0.12 \cdp -0.025 \cpDC -0.11 \ctp +65 \cpG \\
      \hline
      $gg\to H$ ($p_{\textrm T}^{H}>650$~GeV)&
      4 \ctG -0.07 \ctpl1 -0.07 \ctpl2 +0.12 \cdp -0.025 \cpDC -0.12 \ctp +90 \cpG \\
      \hline      
      $gg\to H$ (0-jet, $p_{\textrm T}^{H}<10$~GeV)&
      1.57 \ctG -0.060 \ctpl1 -0.060 \ctpl2 +0.121 \cdp -0.030 \cpDC -0.122 \ctp +38 \cpG \\
      \hline
      $gg\to H$ (0-jet, $p_{\textrm T}^{H}>10$~GeV)&
      1.58 \ctG -0.060 \ctpl1 -0.060 \ctpl2 +0.121 \cdp -0.030 \cpDC -0.121 \ctp +38 \cpG \\
      \hline
      $gg\to H$ (1-jet,$p_{\textrm T}^{H}<60$~GeV)&
      1.59 \ctG -0.060 \ctpl1 -0.060 \ctpl2 +0.121 \cdp -0.030 \cpDC -0.121 \ctp +39.3 \cpG \\
      \hline
      $gg\to H$ (1-jet,$60<p_{\textrm T}^{H}<120$~GeV)&
      1.60 \ctG -0.060 \ctpl1 -0.060 \ctpl2 +0.121 \cdp -0.030 \cpDC -0.121 \ctp +40.6 \cpG \\
      \hline
      $gg\to H$ (1-jet,$120<p_{\textrm T}^{H}<200$~GeV)&
      1.64 \ctG -0.063 \ctpl1 -0.063 \ctpl2 +0.126 \cdp -0.031 \cpDC -0.124 \ctp +43.6 \cpG \\
      \hline
      
       \bottomrule
\end{tabular}
}

\caption{Parametrization of the $gg\to H$ bins with no jet, 0-jet and 1-jet selection of the STXS as defined in its stage 1.2 with the parameters definitions of the \SMEFTatNLO\ model. The numbers are rounded according to their statistical uncertainty.}
\label{tag:higgseft:stxsggh1}
\end{table}




\begin{table}[h!]
 \adjustbox{max width=\textwidth}{
  \begin{tabular}{p{0.40\textwidth} p{0.60\textwidth}}
    \toprule


      \hline
      \textbf{Bin} & Parametrization \\
      \hline
      $gg\to H$ ($\geq$ 2-jet, $m_{\textrm jj}<350$~GeV, $p_{\textrm T}^{H}<60$~GeV)&
      1.62 \ctG -0.061 \ctpl1 -0.061 \ctpl2 +0.126 \cdp -0.031 \cpDC -0.122 \ctp +41 \cpG \\
      \hline
      $gg\to H$ ($\geq$ 2-jet, $m_{\textrm jj}<350$~GeV, $60<p_{\textrm T}^{H}<120$~GeV)&
      +1.63 \ctG -0.061 \ctpl1 -0.061 \ctpl2 +0.120 \cdp -0.031 \cpDC -0.121 \ctp +40.8 \cpG  \\
      \hline
      $gg\to H$ ($\geq$ 2-jet, $m_{\textrm jj}<350$~GeV, $120<p_{\textrm T}^{H}<200$~GeV)&
      +1.69 \ctG -0.062 \ctpl1 -0.062 \ctpl2 +0.120 \cdp -0.030 \cpDC -0.122 \ctp +45 \cpG  \\
      \hline
      $gg\to H$ ($\geq$ 2-jet, $350<m_{\textrm jj}<700$~GeV, $p_{\textrm T}^{H}<200$~GeV, $p_{\textrm T}^{Hjj}<25$~GeV)&
      +1.5 \ctG -0.056 \ctpl1 -0.056 \ctpl2 +0.113 \cdp -0.027 \cpDC -0.113 \ctp +42 \cpG \\
      \hline
      $gg\to H$ ($\geq$ 2-jet, $350<m_{\textrm jj}<700$~GeV, $p_{\textrm T}^{H}<200$~GeV, $p_{\textrm T}^{Hjj}>25$~GeV)&
      
      +1.60 \ctG -0.060 \ctpl1 -0.060 \ctpl2 +0.117 \cdp -0.028 \cpDC -0.126 \ctp + 40 \cpG  \\
      \hline
      $gg\to H$ ($\geq$ 2-jet, $m_{\textrm jj}>700$~GeV, $p_{\textrm T}^{H}<200$~GeV, $p_{\textrm T}^{Hjj}<25$~GeV)&  
      +1.7 \ctG -0.058 \ctpl1 -0.058 \ctpl2 +0.12 \cdp -0.033 \cpDC -0.12 \ctp +48 \cpG  \\
      \hline
      $gg\to H$ ($\geq$ 2-jet, $m_{\textrm jj}>700$~GeV, $p_{\textrm T}^{H}<200$~GeV, $p_{\textrm T}^{Hjj}>25$~GeV)&  
      +1.7 \ctG -0.062 \ctpl1 -0.062 \ctpl2 +0.114 \cdp -0.031 \cpDC -0.118 \ctp +44 \cpG \\
      \hline

       \bottomrule
\end{tabular}
}

\caption{Parametrization of the $gg\to H$ bins with 2 or more jets selection  of the STXS as defined in its stage 1.2 with the parameters definitions of the \SMEFTatNLO\ model. The numbers are rounded according to their statistical uncertainty.}
\label{tag:higgseft:stxsggh2}
\end{table}

    
    
The parametrization of \cpG\ for the $gg\to H$ production mode is different in the \SMEFTsim\ and \SMEFTatNLO\ for 1-jet and 2-jet . It has been checked that for the 0-jet case the values of the inclusive cross section in those models is the same. In this case, the same SMEFT effects are observed. However, when we add jets to the final state, the parametrization changes significantly (it can be compared to the one shown in~\cite{ATL-PHYS-PUB-2019-042}). This is expected due to the different implementation of the process and different diagrams included. In Figure~\ref{fig:higgseft:diagram} is depicted an example diagram which is included in \SMEFTatNLO\ and not considered in \SMEFTsim.

\begin{figure}
  \centering
    \begin{tikzpicture}[node distance=1cm and 1.5cm,scale=0.7]
      \coordinate[label=left:$g$] (q1) at (0,2);
      \coordinate[label=left:$g$] (q2) at (0,-2);
      \coordinate[] (V1) at (1,0);
      \coordinate[] (V2) at (3,0);
      \coordinate[] (V3) at (4.5,2);
      \coordinate[] (V4) at (4.5,0);
      \coordinate[] (V5) at (4.5,-2);
      \coordinate[label=right:$H$] (q3) at (6,2);
      \coordinate[label=right:$g$] (q4) at (6,0);
      \coordinate[label=right:$g$] (q5) at (6,-2);
      \draw[gluon] (q1) -- (V1);
      \draw[gluon] (q2) -- (V1);
      \draw[gluon] (V1) -- node[above]{$g$} (V2);
      \draw[particle] (V2) -- node[left]{$t$} (V3);
      \draw[particle] (V5) -- node[left]{$t$} (V2);
      \draw[particle] (V3) -- node[left]{$t$} (V4);
      \draw[particle] (V4) -- node[left]{$t$} (V5);
      \draw[scalar] (V3) -- (q3);
      \draw[gluon] (V4) -- (q4);
      \draw[gluon] (V5) -- (q5);

    \end{tikzpicture}\\
    \caption{ Example diagram contributing to $gg\to H +j$ which is not considered in \SMEFTsim\ but it is implemented in \SMEFTatNLO.}
    \label{fig:higgseft:diagram}
\end{figure}

\subsection{Summary and conclusions}
In the absence of hints for new physics in the LHC, the SMEFT approach started to be widely adopted by the experimental collaborations for the interpretation of their measurements. In order to be able to have predictions for the SMEFT, implementation of the SM plus dimension-6 Lagrangian in form of UFO files that can be interfaced with modern event generators are needed. Two different tools: \SMEFTsim\ and \SMEFTatNLO\ has been used and compare to study the $gg\to H$ and $gg\to ZH$ loop-induced processes. For the former, which is mainly meant for performing LO calculations, the $gg\to ZH$ process cannot be simulated and only one-loop functions for $gg\to H$ are implemented. It lacks, for example, of $gg\to Hg$ one loop-function exists making the calculation of Higgs plus jets unreliable. It also truncates the Lagrangian for contributions that have a loop supression on top of the $\Lambda^{-2}$.

In this work we have compared both tools for the ttH and ZH production processes. The agreement between the predictions for the SM and interference terms is excellent except for the $\mathcal{O}_{\textrm tG}$ operator. Some other operators like $\mathcal{O}_{\textrm tW}$, $\mathcal{O}_{\textrm tZ}$, or two-fermion currents involving quarks cannot be directly compared. It would be helpful for the user to to have a clear mapping between each operator in both models.

The SMEFT effects have been studied by means of the distortion of the SM prediction shape and normalization in differential cross sections as well as the parametrization of STXS bins. Only the interference effects have been shown. For $gg\to H$
\textcolor{red}{Fill with conclusions of the plots}
The parametrisation in terms of STXS bins for $\mathcal{O}_{\phi G}$ differs from others that can be found in the literature using \SMEFTsim\ due to the differences in the implementation of this process in both tools.
For $gg\to ZH$, with $Z\to l^{+}l^{-}$,  many operators change the cross sections. However, most of them just introduce a deviation in the normalization of the SM predictions at the interference level whithout distorting the SM shape. Among the ones that have an energy dependence we can find: $\mathcal{O}_{tG}$,...

The use of \SMEFTsim\ for EFT interpretations of Higgs measurements can result insatisfactory for Higgs loop-induced processes. In this cases, as we have shown, the \SMEFTatNLO\ tool can be used instead. The NLO effects on the decays have not been studied here. This work could have been extended with studies of $H\to \gamma\gamma$ and $ H\to Z\gamma$. They exist in the literature~\cite{Dawson:2018liq} for $H\to \gamma\gamma$. However, none of the tools are able to provide the NLO QED corrections for these processes in the SMEFT.



~\newline~

%= undefine macros (MANDATORY) ====================
\let\Herwig\undefined
\let\Pythia\undefined
\let\Sherpa\undefined
\let\Rivet\undefined
\let\Professor\undefined
\let\eps\undefined
\let\mc\undefined
\let\mr\undefined
\let\mb\undefined
\let\tm\undefined




